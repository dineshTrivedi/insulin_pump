\section{Proteus \emph{Design Suite}}
Proteus é um suíte, conjunto de \emph{features}, desenvolvido pela Labcenter Eletronics Ltd. É um software para: simulação de microcontroladores, de circuitos eletrônicos e para desenvolvimento de placa de circuito impresso. Os componentes desse sistema são:
\begin{itemize}
\item ISIS \emph{Schematic Capture}: Ferramenta utilizada para se adicionar objetos, componentes para simulação;
\item PROSPICE \emph{Mixed mode SPICE simulation}: Simulador industrial padrão SPICE3F5, combinado com simulador digital de alta velocidade. \emph{Spice, Simulation Program with Integrated Circuit Emphasis}, é um simulador para propósitos gerais para sistemas eletrônicos analógicos.
\item ARES PCB \emph{Layout}: Sistema de PCB, \emph{Printed Circuit Board}, ou melhor, placa de circuito impresso, design de alto desempenho com posicionamento automático de componente, auto-roteamento, entre outras \emph{features} relacionadas ao desenvolvimento de placas de circuito impresso.
\item VSM - \emph{Virtual System Modelling}: Permite simular software embarcado para microcontroladores, disponíveis em suas bibliotecas, ao lado de seu projeto de hardware \cite{proteus2013, wikipedia2012spice}.
\end{itemize}

\subsection{ISIS \emph{Schematic Capture}}
Como já foi dito, essa \emph{feature} se trata da funcionalidade de simulação e montagem de circuitos eletrônicos. Permite que o circuito em questão seja depurado de forma simples e, além disso, possibilita uma integração com códigos desenvolvidos para microcontroladores suportados por ele. Seus componentes são:

\begin{itemize}
\item Circuitos integrados das famílias: 74ALS, 74AS, 74F, 74HC, 74HCT, 74LS, 74S e 74STD;
\item Componentes analógicos;
\item Medidores de corrente e tensão para serem utilizados durante a depuração;
\item Geradores como fontes e \emph{clock}, ambos ajustáveis para o valor desejado;
\item Semicondutores como diodo, transistor, LEDs e outros;
\item Componentes básicos como chaves, resistores, capacitores e indutores;
\item Atuadores como motor DC, motor de passo e outros;
\item Microcontroladores da família 80XXX, AT e PIC;
\item Memórias, \emph{displays}, CMOS e outros.
\end{itemize}

Seu uso é bem simples, pois é como se estivesse desenhando o circuito em um papel. Após ser desenhado é possível utilizar as ferramentas de auxilio para depurar e assim coletar dados como voltagem e corrente do circuito e até mesmo se o funcionamento está conforme o esperado.